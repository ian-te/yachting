% Everything to the right of a  %  is a remark to you and is ignored by LaTeX.

% The Local Guide tells how to run LaTeX.

% WARNING!  Do not type any of the following 10 characters except as directed:
%                &   $   #   %   _   {   }   ^   ~   \   

\documentclass{article}        % Your input file must contain these two lines 
\usepackage{gensymb}
\usepackage[utf8]{inputenc}
\usepackage[russian]{babel}
\begin{document}               % plus the \end{document} command at the end.

\section{Устройство паруса}          
Парус---движитель. Парус делается из парусины, из любого материала (японцы ставили паруса из плексигласса)

\begin{enumerate}
	\item Фаловый угол 
	\item Фаловая дощечка для жёсткости 
	\item Передняя шкаторина
	\item Задняя шкаторина
	\item Нижняя шкаторина
	\item Швы
	\item Латкарманы -- туда вставляются доски, либо длинные куски пластмассы, чтобы парус держал форму
	\item Углы (боуты)
	\item Шкотовый узел
	\item Галзсовый узел
	\item Рифы
	\item Ползуны -- продеваются в ликтроз
\end{enumerate}

\subsection{Виды парусов}
\subsubsection{Прямые паруса}
\subsubsection{Косые паруса}
\begin{itemize}
	\item латинские
	\item гафельные
	\item клевера
	\item стаксели
\end{itemize}

Прямой парус имеет прямоугольную форму или трапецевидную форму, угол между ветром и парусом может составлять 67\degree 

Косые паруса нужны для того, чтоб ходить против ветра


Латинский парус
Парус треугольной формы, привязывают к мачте или рее длинной стороной вдоль диаметральной плоскости судна по направлению к корме и растягивают его при помощи шкота

Гафельный парус
Парус трапецевидной формы, разделяется на трисселя, люгерные, рейковые

Форма
Неправильная трапеция

Крепление
Верхняя часть крепится к гафелю бизани, 
Нижняя часть к бизань гику
Вертикальная сторона крепится к мачте

Клевера
Парус треугольной формы ставят между фок-мачтой и бушпритом


Стаксель
Парус треугольной формы (на яхте он передний).

Генакер
Парус с высоким шкотовым углом, его галсовый угол крепится к палубе на носу яхты или бушприту.

Спинакер
Крепится с помощью спинакер-гика

Мачта состоит из 
Шпора
Верхняя часть мачты называется топ

Бушприт состоит из нока(верхняя часть) и шпор (нижняя часть)

Гик - состоит из пятки и нока


Рея -- горизонтальное рангоутное дерево, подвешенное за середину при помощи бортов и бейфута к мачте или стенге.

Стеньга -- рангоутное дерево, первое удлиннение нижних мачт

Краспица
Одно из основных средств распорка между мачтой стоячего такелажа



Такелаж
Бегучий

Стоячий
Поддерживать и укреплять рангоут судна
К нему относятся тросы трёх типов:
\item ванты
\item таги
\item Ватербак штаги
\item Стеньштаг
\item Топванты
\item Бакштаг
\item Ватерштаг

Бегучий такелаж
Снасти служат для уборки и постановки парусов

К ним относятся
Фалы
Брассы
Шкоты
Галсы
Булини
Гордени
Гитовы
Тапенанты


Рангоутное дерево
Гафель
Реек
Выстрел
Утлегарь
Бушприт
Рей
Бранстеньга


\subsection{Снасти}
Штаг
Стоячий такелаж, ужераживает мачту с носа

Ахтерштаг
Стоячий такелаж, удерживает мачту с корммы

Бакштаг
Стоячий такелаж, удерживает мачту с борта и с кормы

Топенант
Снасть такелажа, поддерживает рангоутное дерево

Фал
Бегучий такелаж, служит для подъема паруса

Нирал
Снасть бегучего такелажа, служит для спуска паруса

Шкот
Снасть бегучего такелажа, служит для управления парусом

Гитов
Бегучий такелаж, по направлению действия противоположный шкоту

Ванта
Стоячий такелаж, удерживает мачту в плоскости шпангоута

Галс
Бегучий такелаж, крепит нижний угол косого паруса

Нирал
Бегучий такелаж, направление которого противоположно фалу (для спуска паруса)



\subsection{Курсы парусов}
Силы, действующий на яхту
Сила тяги

кренящая сила

Сопротивление воздуха

Сила дрейфа

Сила сопротивления воды

Сила бокового сопротивления

Сила направления движения


Мидльшпангоут
Шпангоут, делящий судно пополам
Яхта двигается за счет разницы сил бокового сопротивления и аэродинамической силы


Центр парусности
На пересечении биссектрис проведенных из углов паруса

Центр бокового сопротивления
Может меняться в зависимости от осадки	

Курсы относительно ветра

\begin{itemize}
\item Левентик
	0 - 11.25 
\item Фордевинд
	180 
\item Галфвинд
	Ровно 90 градусов
	270
\item Бейдевинд
	крутой 11.25 - 56.25
	полный 67.5 - 78.75
	полный правый 281.25 - 292.5
	правый крутой 303.75 - 340.75
\item Лавировка
\item Багштаг
	крутой 101.25 - 123.75
	полный 135 - 168.75
	полный правым галсом 191.25 - 225
	крутой правым галсом 236.25 - 258.75
\end{itemize}

В какой борт дует ветер, тем галсом и идем.

\subsubsection{Повороты}
Оверштаг - штагом пересекаем ветер
Фордевинд - кормой пересекаем ветер
Коровий - в сильный ветер и открытый океан -- наиболее безопасный поворот

\subsubsection{Основные типы парусных судов}
Корабль(фрегат)
Три мачты
Полное парусное вооржуение
Все паруса прямые
На бизань мачте косой парус

Барк
Три мачты
Бизань - 2-3 косых паруса

Бриг 
Две мачты
Прямые паруса и косой парус на гроте

Баркентина
три мачты
На фоке прямые, все остальные паруса косые

Бригантина
Две мачты
На фоке прямые и косые на грот-мачте.

Гафельная шхуна
Две мачты с гафельным вооружением

Марсельская шхуна
Двух-трёхмачтовые
Гафельная + прямые паруса



Одномачтовые суда
Кэт
Шлюп
Тендер

Двухмачтовые суда
Кеч --- задняя мачта ниже передней, передняя мачта - грот, задняя - бизань. Паруса -- косые. Бизань мачта стоит впереди балер руля
Иол --- бизань находится в корму от балер руля 
Шхуна
Бриг
Бригантина

Трёхмачтовые суда
Барк
Баркентина
Корабль
Фрегат
Шхуна

\subsection{15 основных морских узлов}
Беседочный Король узлов (булинь)
Португальский булинь Португальский бесеточный (?)
Восьмёрка
Кинжальный
Прямой (Геркулесовый или Геракловый)
Фламанская петля
Фламанский узел
Фаловый
Простой штык с двумя шлагами
Удавка
Рыбацкий штык
Эшафотный (висельный)
Стопорный
Рифовый
Прямой (для связывания тросов различной величины)

Посмотреть устройство судна

\section{Лекция 3: Теория и устройство судна}
Признаки и классификации судов

по назначению
по району плаванья
по способу движения
По типу главного двигателя
по способу поддержания на воде
по типу движителя
по материалу корпуса
по архитектурно-конструктивному типу
по количеству гребных валов

Мореходные навигационные качества судов
плавучесть
остойчивость
непотопляемость
ходкость
управляемость


Судно состоит из шпангоутов и набора корпуса

Набор корпуса зависит от того, для чего строится судно

Судно состоит из поперечных балок и продольных балок. Набор бывает поперечный, продольный или смешанный

Плоскости
Основная
Диаметральная
Плоскость шпангоута

Судно имеет три вида:
вид сбоку
вид сверху (полуширота)
вид с носа или кормы

На теоретический чертеж наносятся шпангоуты
(до 20)

Расстояние между шпангоутами называется шпация


Флора (флор) --- нижняя часть горизонтального набора корпуса, 
	стягивающая нижнюю часть шпангоута, или же основная
	днищевая поперечная балка, нижняя часть шпангоутной рамы

Бимс --- балка поперечного набора таврового профиля, связывающая 
	бортовые ветви шпангоута, и поддерживающая палубу.


Ширстрек --- пояс бортовой обшивки, примыкающий к верхней непрерывной 
	палубе судна, основная продольная связь.

Стрингер --- усиленная продольная балка набора корпуса. Бывает
	днищевой, скуловой, бортовой и палубный (поперечная - на дельте)

Кница --- пластина треугольной или трапецевидной формы, соединяет детали корпуса под углом друг к другу

Шпангоут --- криволинейная поперечная балка

Голубица --- небольшой вырез, в нижней части флора, для протока воды или прокладки коммуникаций.

Кильсон --- днищевая продольная балка -- киль судна.

Привальный брус --- элемент продольного набора (для того, чтобы предохранить корпус от повреждений
	при швартовке. 

Пайолы --- пол. Под ним скапливаются подсланивые воды.

Подсланивые воды

Фекальные воды

Форштевень --- передняя часть судна.

Ахтерштевень --- задняя часть судна

Все вертикальные стойки на всех судах называются пиллерс, а все пороги -- комминкс




\subsection{Мореходные качества судна}
Остойчивость --- способность судна, отклоненное внешними силами от положения равновесия
	и предоставленное самому себе возвращаться в положение равновесия.

Остойчивость бывает статистическая и динамическая.

Плавучесть --- способность судна ходить при заданной нагрузке, имея заданную осадку.
Мерой плавучести служит водоизмещение

Запас плавучести --- Объем водонепроницаемых переборок выше ватерлинии

Ходкость --- способность судна поддерживать скорость хода и маневренность на заданных курсах

Качка --- бывает бортовая и килевая.

Управляемость характеризуется диаметром циркуляции и как судно слушается руля.

Курсовая устойчивость --- лежание на курсе без внешнего вмешательства

Активное торможение (реверс) --- 

/subsubsection{Пять способов (случаев) затопления}
\begin{itemize}
	\item Полностью затоплен отсек, или несколько симметричных отсеков полностью под ватерлинией относительно диаметральной плоскости корабля
	\item Полностью затоплен отсек несимметричной диаметральной плоскости судна. Отсек находится ниже ватерлинии. В этом случае уменьшается метацентрическая высота, и получается угол заката со стороны затопленного борта
	\item Полностью затоплен отсек несимметричной диаметральной плоскости, с креном, противоположному затопленному отсеку, при этом уменьшается динамическая и статическая остойчивость
\end{itemize}

Каждое судно имеет угол заката. 
Угол заката --- тот момент, когда судно не может вернуться в исходное положение. После этого судно переворачивается, и делает оверкиль -- переворот судна кверху килем.
Поэтому в каждом судне висит кренометр.






\end{document}                 % The input file ends with this command.

