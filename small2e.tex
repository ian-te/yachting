% Everything to the right of a  %  is a remark to you and is ignored by LaTeX.

% The Local Guide tells how to run LaTeX.

% WARNING!  Do not type any of the following 10 characters except as directed:
%                &   $   #   %   _   {   }   ^   ~   \   

\documentclass{article}        % Your input file must contain these two lines 
\usepackage[utf8]{inputenc}
\usepackage[russian]{babel}
\begin{document}               % plus the \end{document} command at the end.




\section{Устройство паруса}          
Парус --- движитель. Парус делается из парусины, из любого материала (японцы ставили паруса из плексигласса)

Устройство паруса
\begin{itemize}
	\item Фаловый угол 
	\item Фаловая дощечка для жёсткости 
	\item Передняя шкаторина
	\item Задняя шкаторина
	\item Нижняя шкаторина
	\item Швы
	\item Латкарманы -- туда вставляются доски, либо длинные куски пластмассы, чтобы парус держал форму
	\item Углы (боуты)
	\item Шкотовый узел
	\item Галзсовый узел
	\item Рифы
	\item Ползуны -- продеваются в ликтроз
\end{itemize}

\subsection{Виды парусов}
\subsubsection{Прямые паруса}
\subsubsection{Косые паруса}
\begin{itemize}
	\item латинские
	\item гафельные
	\item клевера
	\item стаксели
\end{itemize}

Прямой парус имеет прямоугольную форму или трапецевидную форму, угол между ветром и парусом может составлять $67\deg$

Косые паруса нужны для того, чтоб ходить против ветра

Латинский парус
Парус треугольной формы, привязывают к мачте или рее длинной стороной вдоль диаметральной плоскости судна по направлению к корме и растягивают его при помощи шкота

Гафельный парус
Парус трапецевидной формы, разделяется на трисселя, люгерные, рейковые

Форма
Неправильная трапеция

Крепление
Верхняя часть крепится к гафелю бизани, 
Нижняя часть к бизань гику
Вертикальная сторона крепится к мачте

Клевера
Парус треугольной формы ставят между фок-мачтой и бушпритом


Стаксель
Парус треугольной формы (на яхте он передний).

Генакер
Парус с высоким шкотовым углом, его галсовый угол крепится к палубе на носу яхты или бушприту.

Спинакер
Крепится с помощью спинакер-гика

Мачта состоит из 
Шпора
Верхняя часть мачты называется топ

Бушприт состоит из нока(верхняя часть) и шпор (нижняя часть)

Гик - состоит из пятки и нока


Рея -- горизонтальное рангоутное дерево, подвешенное за середину при помощи бортов и бейфута к мачте или стенге.

Стеньга -- рангоутное дерево, первое удлиннение нижних мачт

Краспица
Одно из основных средств распорка между мачтой стоячего такелажа



Такелаж
Бегучий

Стоячий
Поддерживать и укреплять рангоут судна
К нему относятся тросы трёх типов:
ванты
Штаги
Ватербак штаги
Стеньштаг
Топванты
Бакштаг
Ватерштаг

Бегучий такелаж
Снасти служат для уборки и постановки парусов

К ним относятся
Фалы
Брассы
Шкоты
Галсы
Булини
Гордени
Гитовы
Тапенанты


Рангоутное дерево
Гафель
Реек
Выстрел
Утлегарь
Бушприт
Рей
Бранстеньга


\subsection{Снасти}
Штаг
Стоячий такелаж, ужераживает мачту с носа

Ахтерштаг
Стоячий такелаж, удерживает мачту с корммы

Бакштаг
Стоячий такелаж, удерживает мачту с борта и с кормы

Топенант
Снасть такелажа, поддерживает рангоутное дерево

Фал
Бегучий такелаж, служит для подъема паруса

Нирал
Снасть бегучего такелажа, служит для спуска паруса

Шкот
Снасть бегучего такелажа, служит для управления парусом

Гитов
Бегучий такелаж, по направлению действия противоположный шкоту

Ванта
Стоячий такелаж, удерживает мачту в плоскости шпангоута

Галс
Бегучий такелаж, крепит нижний угол косого паруса

Нирал
Бегучий такелаж, направление которого противоположно фалу (для спуска паруса)



\subsection{Курсы парусов}
Силы, действующий на яхту
Сила тяги

кренящая сила

Сопротивление воздуха

Сила дрейфа

Сила сопротивления воды

Сила бокового сопротивления

Сила направления движения


Мидльшпангоут
Шпангоут, делящий судно пополам
Яхта двигается за счет разницы сил бокового сопротивления и аэродинамической силы


Центр парусности
На пересечении биссектрис проведенных из углов паруса

Центр бокового сопротивления
Может меняться в зависимости от осадки

Курсы относительно ветра
Левентик
	0 - 11.25 \textdegree
Фордевинд
	180 
Галфвинд
	Ровно 90 градусов
	270
Бейдевинд
	крутой 11.25 - 56.25
	полный 67.5 - 78.75
	полный правый 281.25 - 292.5
	правый крутой 303.75 - 340.75
Лавировка
Багштаг
	крутой 101.25 - 123.75
	полный 135 - 168.75
	полный правым галсом 191.25 - 225
	крутой правым галсом 236.25 - 258.75


В какой борт дует ветер, тем галсом и идем.

\subsubsection{Повороты}
Оверштаг - штагом пересекаем ветер
Фордевинд - кормой пересекаем ветер
Коровий - в сильный ветер и открытый океан -- наиболее безопасный поворот

\subsubsection{Основные типы парусных судов}
Корабль(фрегат)
Три мачты
Полное парусное вооржуение
Все паруса прямые
На бизань мачте косой парус

Барк
Три мачты
Бизань - 2-3 косых паруса

Бриг 
Две мачты
Прямые паруса и косой парус на гроте

Баркентина
три мачты
На фоке прямые, все остальные паруса косые

Бригантина
Две мачты
На фоке прямые и косые на грот-мачте.

Гафельная шхуна
Две мачты с гафельным вооружением

Марсельская шхуна
Двух-трёхмачтовые
Гафельная + прямые паруса



Одномачтовые суда
Кэт
Шлюп
Тендер

Двухмачтовые суда
Кеч --- задняя мачта ниже передней, передняя мачта - грот, задняя - бизань. Паруса -- косые. Бизань мачта стоит впереди балер руля
Иол --- бизань находится в корму от балер руля 
Шхуна
Бриг
Бригантина

Трёхмачтовые суда
Барк
Баркентина
Корабль
Фрегат
Шхуна

\subsection{15 основных морских узлов}
Беседочный Король узлов (булинь)
Португальский булинь Португальский бесеточный (?)
Восьмёрка
Кинжальный
Прямой (Геркулесовый или Геракловый)
Фламанская петля
Фламанский узел
Фаловый
Простой штык с двумя шлагами
Удавка
Рыбацкий штык
Эшафотный (висельный)
Стопорный
Рифовый
Прямой (для связывания тросов различной величины)

Посмотреть устройство судна

\end{document}                 % The input file ends with this command.

