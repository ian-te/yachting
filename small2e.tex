% Everything to the right of a  %  is a remark to you and is ignored by LaTeX.

% The Local Guide tells how to run LaTeX.

% WARNING!  Do not type any of the following 10 characters except as directed:
%                &   $   #   %   _   {   }   ^   ~   \   

\documentclass{article}        % Your input file must contain these two lines 
\usepackage{gensymb}
\usepackage[utf8]{inputenc}
\usepackage[russian]{babel}
\begin{document}               % plus the \end{document} command at the end.

\section{Устройство паруса}          
Парус---движитель. Парус делается из парусины, из любого материала (японцы ставили паруса из плексигласса)

\begin{enumerate}
	\item Фаловый угол 
	\item Фаловая дощечка для жёсткости 
	\item Передняя шкаторина
	\item Задняя шкаторина
	\item Нижняя шкаторина
	\item Швы
	\item Латкарманы -- туда вставляются доски, либо длинные куски пластмассы, чтобы парус держал форму
	\item Углы (боуты)
	\item Шкотовый узел
	\item Галзсовый узел
	\item Рифы
	\item Ползуны -- продеваются в ликтроз
\end{enumerate}

\subsection{Виды парусов}
\subsubsection{Прямые паруса}
\subsubsection{Косые паруса}
\begin{itemize}
	\item латинские
\item гафельные
	\item клевера
	\item стаксели
\end{itemize}

Прямой парус имеет прямоугольную форму или трапецевидную форму, угол между ветром и парусом может составлять 67\degree 

Косые паруса нужны для того, чтоб ходить против ветра


Латинский парус
Парус треугольной формы, привязывают к мачте или рее длинной стороной вдоль диаметральной плоскости судна по направлению к корме и растягивают его при помощи шкота

Гафельный парус
Парус трапецевидной формы, разделяется на трисселя, люгерные, рейковые

Форма
Неправильная трапеция

Крепление
Верхняя часть крепится к гафелю бизани, 
Нижняя часть к бизань гику
Вертикальная сторона крепится к мачте

Клевера
Парус треугольной формы ставят между фок-мачтой и бушпритом


Стаксель
Парус треугольной формы (на яхте он передний).

Генакер
Парус с высоким шкотовым углом, его галсовый угол крепится к палубе на носу яхты или бушприту.

Спинакер
Крепится с помощью спинакер-гика

Мачта состоит из 
Шпора
Верхняя часть мачты называется топ

Бушприт состоит из нока(верхняя часть) и шпор (нижняя часть)

Гик - состоит из пятки и нока


Рея -- горизонтальное рангоутное дерево, подвешенное за середину при помощи бортов и бейфута к мачте или стенге.

Стеньга -- рангоутное дерево, первое удлиннение нижних мачт

Краспица
Одно из основных средств распорка между мачтой стоячего такелажа



Такелаж
Бегучий

Стоячий
Поддерживать и укреплять рангоут судна
К нему относятся тросы трёх типов:
\begin{itemize}
\item ванты
\item таги
\item Ватербак штаги
\item Стеньштаг
\item Топванты
\item Бакштаг
\item Ватерштаг
\end{itemize}

Бегучий такелаж
Снасти служат для уборки и постановки парусов

К ним относятся
Фалы
Брассы
Шкоты
Галсы
Булини
Гордени
Гитовы
Тапенанты


Рангоутное дерево
Гафель
Реек
Выстрел
Утлегарь
Бушприт
Рей
Бранстеньга

\subsection{Снасти}
\begin{description}
	\item[Штаг] Стоячий такелаж, ужераживает мачту с носа
	\item[ Ахтерштаг ] Стоячий такелаж, удерживает мачту с корммы
	\item[ Бакштаг ] Стоячий такелаж, удерживает мачту с борта и с кормы
	\item[ Топенант ] Снасть такелажа, поддерживает рангоутное дерево

	\item [ Фал ] Бегучий такелаж, служит для подъема паруса

	\item [ Нирал ] Снасть бегучего такелажа, служит для спуска паруса 
	\item [ Шкот ] Снасть бегучего такелажа, служит для управления парусом

	\item [ Гитов ] Бегучий такелаж, по направлению действия противоположный шкоту

	\item [ Ванта ] Стоячий такелаж, удерживает мачту в плоскости шпангоута

	\item [ Галс ] Бегучий такелаж, крепит нижний угол косого паруса

	\item [ Нирал ] Бегучий такелаж, направление которого противоположно фалу (для спуска паруса)

\end{definition}


\subsection{Курсы парусов}
Силы, действующий на яхту
Сила тяги

кренящая сила

Сопротивление воздуха

Сила дрейфа

Сила сопротивления воды

Сила бокового сопротивления

Сила направления движения


Мидльшпангоут
Шпангоут, делящий судно пополам
Яхта двигается за счет разницы сил бокового сопротивления и аэродинамической силы


Центр парусности
На пересечении биссектрис проведенных из углов паруса

Центр бокового сопротивления
Может меняться в зависимости от осадки	

Курсы относительноsветрs

\begin{itemize}
\item Левентик
	0 - 11.25 
\item Фордевинд
	180 
\item Галфвинд
	Ровно 90 градусов
	270
\item Бейдевинд
	крутой 11.25 - 56.25
	полный 67.5 - 78.75
	полный правый 281.25 - 292.5
	правый крутой 303.75 - 340.75
\item Лавировка
\item Багштаг
	крутой 101.25 - 123.75
	полный 135 - 168.75
	полный правым галсом 191.25 - 225
	крутой правым галсом 236.25 - 258.75
\end{itemize}

В какой борт дует ветер, тем галсом и идем.

\subsubsection{Повороты}
Оверштаг - штагом пересекаем ветер
Фордевинд - кормой пересекаем ветер
Коровий - в сильный ветер и открытый океан -- наиболее безопасный поворот

\subsubsection{Основные типы парусных судов}
Корабль(фрегат)
Три мачты
Полное парусное вооржуение
Все паруса прямые
На бизань мачте косой парус

Барк
Три мачты
Бизань - 2-3 косых паруса

Бриг 
Две мачты
Прямые паруса и косой парус на гроте

Баркентина
три мачты
На фоке прямые, все остальные паруса косые

Бригантина
Две мачты
На фоке прямые и косые на грот-мачте.

Гафельная шхуна
Две мачты с гафельным вооружением

Марсельская шхуна
Двух-трёхмачтовые
Гафельная + прямые паруса



Одномачтовые суда
Кэт
Шлюп
Тендер

Двухмачтовые суда
Кеч --- задняя мачта ниже передней, передняя мачта - грот, задняя - бизань. Паруса -- косые. Бизань мачта стоит впереди балер руля
Иол --- бизань находится в корму от балер руля 
Шхуна
Бриг
Бригантина

Трёхмачтовые суда
Барк
Баркентина
Корабль
Фрегат
Шхуна

\subsection{15 основных морских узлов}
\begin{itemize}
	\item Беседочный Король узлов (булинь)
	\item Португальский булинь  Португальский бесеточный (?) 
	\item Восьмёрка
	\item Кинжальный
	\item Прямой (Геркулесовый или Геракловый)
	\item Фламанская петля
	\item Фламанский узел
	\item Фаловый
	\item Простой штык с двумя шлагами
	\item Удавка
	\item Рыбацкий штык
	\item Эшафотный (висельный)
	\item Стопорный
	\item Рифовый
	\item Прямой (для связывания тросов различной величины)
\end{itemize}
Посмотреть устройство судна

\section{Лекция 3: Теория и устройство судна}
Признаки и классификации судов

по назначению
по району плаванья
по способу движения
По типу главного двигателя
по способу поддержания на воде
по типу движителя
по материалу корпуса
по архитектурно-конструктивному типу
по количеству гребных валов

Мореходные навигационные качества судов
плавучесть
остойчивость
непотопляемость
ходкость
управляемость


Судно состоит из шпангоутов и набора корпуса

Набор корпуса зависит от того, для чего строится судно

Судно состоит из поперечных балок и продольных балок. Набор бывает поперечный, продольный или смешанный

Плоскости
Основная
Диаметральная
Плоскость шпангоута

Судно имеет три вида:
вид сбоку
вид сверху (полуширота)
вид с носа или кормы

На теоретический чертеж наносятся шпангоуты
(до 20)

Расстояние между шпангоутами называется шпация


Флора (флор) --- нижняя часть горизонтального набора корпуса, 
	стягивающая нижнюю часть шпангоута, или же основная
	днищевая поперечная балка, нижняя часть шпангоутной рамы

Бимс --- балка поперечного набора таврового профиля, связывающая 
	бортовые ветви шпангоута, и поддерживающая палубу.


Ширстрек --- пояс бортовой обшивки, примыкающий к верхней непрерывной 
	палубе судна, основная продольная связь.

Стрингер --- усиленная продольная балка набора корпуса. Бывает
	днищевой, скуловой, бортовой и палубный (поперечная - на дельте)

Кница --- пластина треугольной или трапецевидной формы, соединяет детали корпуса под углом друг к другу

Шпангоут --- криволинейная поперечная балка

Голубица --- небольшой вырез, в нижней части флора, для протока воды или прокладки коммуникаций.

Кильсон --- днищевая продольная балка -- киль судна.

Привальный брус --- элемент продольного набора (для того, чтобы предохранить корпус от повреждений
	при швартовке. 

Пайолы --- пол. Под ним скапливаются подсланивые воды.

Подсланивые воды

Фекальные воды

Форштевень --- передняя часть судна.

Ахтерштевень --- задняя часть судна

Все вертикальные стойки на всех судах называются пиллерс, а все пороги -- комминкс




\subsection{Мореходные качества судна}
Остойчивость --- способность судна, отклоненное внешними силами от положения равновесия
	и предоставленное самому себе возвращаться в положение равновесия.

Остойчивость бывает статистическая и динамическая.

Плавучесть --- способность судна ходить при заданной нагрузке, имея заданную осадку.
Мерой плавучести служит водоизмещение

Запас плавучести --- Объем водонепроницаемых переборок выше ватерлинии

Ходкость --- способность судна поддерживать скорость хода и маневренность на заданных курсах

Качка --- бывает бортовая и килевая.

Управляемость характеризуется диаметром циркуляции и как судно слушается руля.

Курсовая устойчивость --- лежание на курсе без внешнего вмешательства

Активное торможение (реверс) --- 

\subsubsection{Пять способов (случаев) затопления}
\begin{enumerate}
	\item Полностью затоплен отсек, или несколько симметричных отсеков полностью под ватерлинией относительно диаметральной плоскости корабля
	\item Полностью затоплен отсек несимметричной диаметральной плоскости судна. Отсек находится ниже ватерлинии. В этом случае уменьшается метацентрическая высота, и получается угол заката со стороны затопленного борта
	\item Полностью затоплен отсек несимметричной диаметральной плоскости, с креном, противоположному затопленному отсеку, при этом уменьшается динамическая и статическая остойчивость
	\item Судно имет частично затопленный отсек и существует свободная поверхность. Уменьшается метацентр, начальная остойчивость отрицательна. Снижается запас динамической остойчивости.
	\item Когда судно имеет частично затопленный отсек, противоположный затопленному.
\end{enumerate}

Каждое судно имеет угол заката. 
Угол заката --- тот момент, когда судно не может вернуться в исходное положение. После этого судно переворачивается, и делает оверкиль -- переворот судна кверху килем.
Поэтому в каждом судне висит кренометр.

\subsubsection{Метацентрическая высота и центр тяжести}
Метацентрическая высота --- точка пересечения линий сил плавучести в прямом и наклоённом положениях корабля, 
	является поперечным или продольным метацентром. При малых углах наклонения и постоянных водоизмещениях метацентры занимают определенные и постоянные положения. Запрещается эксплуатация судов при возвышении метацентра над центром тяжести менее 0.2м. 

Центр величины (Цв) --- Точка приведения сил плавучести или центр водоизмещения тела.

Центр тяжести --- точка приложения веса судна.

Бонжан --- в 19 веке вывел формулу расчёта плавучести и непотопляемости на диаграмму.
	Называется она масштаб Бонжана

По масштабу Бонжана определяется максимальное плечо остойчивости, угол максимальной диаграммы, начальная метацентрическая высота и угол заката.

Лекция: сведения из теории корабля

\subsection{Основные весовые характеристики судна}
Дедвейт --- сумма масс переменных грузов в тоннах, весь вес судна

Коэффициент полноты площади ватерлинии --- определяется длинной по ватерлинии, площадью фактической ватерлинии, шириной по ватерлинии на мидльшпангоуте.

Коэффициент полноты мидльшпангоута --- определяется шириной по ватерлинии на мидльшпангоуте,
осадкой корпуса на мидльшпангоуте, фактической площадью подводной части мидльшпангоута. 

Коэффициент общей полноты судна --- определяется осадкой корпуса, длиной по ватерлинии, 
	шириной по ватерлинии, объемным водоизмещением

Линейные характеристики судна.
	осадка корпуса
	длина по ватерлинии
	минимальная высота надводного борта

Весовые характеристики судна
	дедвейт
	вес порожним
	грузоподъемность
	весовое водоизмещение

Объемные характеристики судна
	грузовместимость
	запас плавучести
	объемное водоизмещение

\subsubsection{Маневренные характеристики судна}
Рули бывают балансирные, небелансирные и полубалансирные.

Руль вешается в зависимости от того, какая нужна управляемость.

Ставятся либо один, либо два руля.


\subsection{Якоря, якорь-цепи, якорные устройства}
\begin{itemize}
	\item Адмиралтейский
	\item Якорь Холла
	\item Якорь Матросова
	\item Мертвый якорь
	\item Якорь плуг
	\item Якорь Брюса
	\item Ледовый якорь
	\item Плавучий якорь 
		нужен для замедления хода, удержания яхты на курсе, в случаях, когда постановка на стационарный якорь невозможна.
\end{itemize}

Все они состоят из:
	Шеймы
	Штока
	Веретена
	Рога
	Тренда

Все они крепятся к якорь-тросу, либо к якорь-цепи, которая входит в якорный клюс, затем идёт в цепной ящик, или полуклюс, где при помощи жвакогался (скобы) крепится к корпусу судна

При постановке на якорь, якорь-цепь или якорь-канат, кладётся три глубины моря.
Якорь отдаётся всегда на заднем ходу.

На яхтах используется кошка, или якорь Матросова

Поднимаются и опускаются якоря с помощью шпиля и брашпиля

Шпиль --- вертикально расположенное барабанное устройство, для подъема якорей, или для проведения каких-либо тягловых работ с тросами.

Брашпиль --- горизонтально расположенное вертикальное устройство, предназначенное для подъема якорей, или проведения каких-либо тяговых работ с тросамиe


\section{Лекция 4}
\subsection{Морские узлы}
\subsubsection{Узлы, отдаваемые под нагрузкой}
	Стопорные узлы
	Штык простой
	Штык со шлагом
	Рифовый
	Штык с обносом
	Удавка со шлагом

Узлы, которые не могут быть отданы под нагрузкой
	Прямой
	Шкотовый
	Бесеточный
	Выбленочный
	Сваечный
	Буйрепный
	Восьмерка
	Брандшкотовый
	Рыбацкий штык
	Слежной штык

Узлы, являющиеся незатягивающимися петлями
	Беседочный
	Рыбацкий штык

Якорь крепится к якорь канату рыбацким штыком

Узлы, которые можно завязать вокруг предмета для поднятия его на высоту
	Сваечный
	Выбленочный
	Сдвижной штык
	Штык с обносом
	Удавка со шлагом

Узлы бывают
	для утолщения тросов
		Простой
		Кровавый
		Восьмёрка
		Стивидорный
		Устричный
		Пожарная лестница
		Многократная восьмёрка
	незатягивающиеся узлы
	затягивающиеся узлы
	затягивающиеся петли
	
Марка---

г
Дельные вещи (Такелажное дело)
Для работы с тросами применяется такелажный инструмент. 


	Деревянная свайка
	Стальная свайка
	Драёк
	Мушкель
	Киянка
	Полумушкель
	Такелажная лопатка
	Марочница
	Машинка для слома троса
	Такелажные тиски
	Берда
	Трепало
	Гардама
	Парусная игла

Для швартовки используется приспособления, которые называются
	Битинг
	Кнехт
	Полуклюс
	Обух
	Рым
	Утка
	Такелажная скоба

Концы, которые отдаются при швартовке
	Новосой продольный
	Носовой прижимной
	Носовй шпринг
	Кормовой шпринг
	Кормовой прижимной
	оормовой продольный

Существуют постановки на якорь
	на гусёк - постановка на два якоря
		нельзя становиться при сильных приливно-отливных течениях
	постановка на фертоинг


Яхтой всегда движет вымпельный ветер
Складывается из направления откуда дует ветер и направления движения судна
Ветер всегда в компас - течение из компаса

Если дует в лицо -- то на ветру
Если в спину -- под ветром

Глубина измеряется
	эхолотом
	ручным лотом
	футштоком

Ручной лот состоит из конца, на котором размещены марки или с топориками, или с зубцами
На конце его находится свинцовая чушка, изнутри полуполая и смазанная салом или талотом

Футштоки бывают 3-5 м длиной. Это рейка с делением на футы и дюймы. В переводе на русский это рейка, которая окрашивается черным и белым каждые 10см

Якорные цепи маркируются марками и окрашивается.
20м - одно красное звено с маркой
40м - два красных звена с маркой
60м - три красных
80м - 4
100м - 5
120м - одно белое звено с маркой





Управление судном

\section{Радиосвязь}

\subsection{Районы радиосвязи}
А1
А2
А3
А4
\subsection{Водные бассейны}
Внутренние водные бассейны включая участки с с морским режимом судоходства, классифицируются разрядами 
Л Р О М в зависимости от их ветрового режима (в зависимости от высоты волны)
Л -- до 0.6м
Р -- до 1.2м
М -- до 3м
О -- 2м

\subsection{Огни и знаки судов}
Огни и знаки судов и плотов несутся огни и знаки следующей цветоовой характеристики
Белый
Красный
Зеленый
Желтый
Синий
Проблесковый желтый
Проблесковый синий

Огни расположены
в носу
по бортам
на корме
на мачтах


225\degree
топовый красный
225\degree
бортовой зеленый правый борт 112.5\degree
бортовой красный левый борт 112.5\degree
кормовой 135\degree
Буксироовчный желтый 135\degree
Круговой белый 360\degree
Круговой красный 360\degree 225\degree
топовый красный
225\degree
бортовой зеленый правый борт 112.5\degree
бортовой красный левый борт 112.5\degrees
кормовой 135\degrees
Буксироовчный желтый 135\degree
Круговой белый 360\degree
Круговой красный 360\degree
Круговой зеленый 360\degree
Проблескоый желтый 360\degree
Проблесковый синий 360\degree
Синий топовый 360\degree

Речные суда помимо этого несут светоимпульсные отмашки

Носовая отмашка левого борта от траверса судна к носу

Кормовая отмашка левого бора от траверса судна к носу

от травеса к коме 112.5\degree
правого борта 112.5\degree от траверса к носу
правого борта 112.5\degree от траверса к корме

Кроме этого, суда несут стояночные огни
Якорный огонь белого цвета 360\degree



\subsection{Чтение морских карт}
Морские карты -- чтение и обозначени
Характеристика нав. огней и их обозначение

\begin{itemize}
\item П -- постоянный
\item Пр -- проблесковый периодически повторяющиеся проблески 0.5с с затемнением до 2.7 - 3c
\item Пр(2) -- двухпроблесоквый. Проблески 0.5-0.7с, затемнение 3с.
\item Ч -- частопроблесковый, непрерывно повторяющийся -- частые проблески 0.5-0.7с
\item Гр Ч -- группа частопроблесковый. Группа повторяющийхся 4-5 проблесков, с затемнениями 3-3.5с
\item Пу -- пульсирующий непрерывно. Световые импульсы, частота 8-10 импульсов по 1с.
\item ПуПр -- Прерывисто пульсирующий. Группа из 4-6 световых импульсовпо 0.5-0.6с и затемнение 3.5с
\item ЗТМ -- затмевающийся, переодически повторяющиеся проблески длительностью 2.7-3с с кратковременным затемнением длительностью 0.6-0.8с

\end{itemize}
МАМС -- международная ассоциация маячных служб, которая входит в ИМО.

Существуют две системы -- латеральная и координальная, которые приняты
Подразделяются на регион А и регион Б. Мы находимся в регионе А

Регион Б - Америка (и Северная, и Центральная, и Южная), Филипины, Корея, Япония

Координальная система ставится на море
Латеральная система ставится на ВВП стран. 

Знак жёлто-чёрный

Северный мигает постоянно.
Потом часовое правило. 3 - восток, 6 - юг, 9 - запад.

Латеральные знаки
Латеральная система бокового оборудования --- система навигационного 
ограждения участков водной поверхности или объекта, представляющих опасность 
для плавания

Одна из двух наряду с кардинальной системой плавучего ограждения морских или речных опасностей

Латеральная система используется как правило для ограждения 
продольных судовых ходов, имеющиюх ярко выраженные стороны.

К таким судовым ходам можно отнести 
	фарватеры
	морские и речные каналы
	полосу судового движения на реках


Знаки бывают
	запрещающие
	предупреждающие
	предписывающие
	указательные
	информационные

Ставятся они на воде и на берегу

В качестве предостерегающих знаков используются
	бакины
	буи
	вехель

А также береговые ограждения в виде
	маяков
	створов
	береговых знаков

Движение по латеральной системе осуществляется либо между предостерегающими знаками, 
либо вдоль осевой линии по которой стоят знаки

Стороны фарватера однозначно привязываются либо к направлению течения на реках, либо к направлению следования с моря


Следует различать окраску знаков латеральной системы на ВВП России (СССР) и на море.
На ВВП левый по течению стороне присвоен белый цвет, правой --- красный цвет.

На морских навигационных картах обязательно указывается к какому типу относится данный район.
Регион А -- красный слева

МАМС для всего мира принял два типа латеральной системы, 
которые отличаются противоположной цветовой кодировкой.

Латеральнгая система А -- красный слева
Левая сторона -- знаки красного цвета, бакены цилиндрической прямоугольной формы
Правая сторона -- знаки белого, черного или зеленого цвета, бакены конусообразной или треугольной формы, используются в Европе, в России, Австралии, Африке, приемущественно в Азии за исключением Филипин, Кореи и Японии


Латеральная система Б -- красный справа
Цвета полностью противополжны системе А.

В России, относящейся к региону А, цветое обозначение сторон в морских устьях рек 
имеет следующий смысл. При движении с моря красный --- слева
а при движении по реке --- речная система. Красный --- справа

Границы между морской и речной системой ограждения указываются на картах и в лоциях.

Существуют разделительные(всегда вертикальная полоса) и поворотные (горизонтальная полоса)


\subsection{Реки}

Динамическая ось речного потока называется стрежнем реки.

Прижимное течение --- течение, направленное к вогнутому берегу

Свальное течение --- течение, направленное под углом к судовому ходу

Беспорядочное вращательное движение воды в виде подвижных вихрей называется Майдан, или Бардак.

Равные скорости течения на реке называются Изотахи. Наибольшная неравномерность распределения скоростей течений по ширине реки наблюдается на поворотах русла.

Судно во время движения узким фарватеорм или проходом, должно держаться всегда внешней границы фарватера

Судно идущее левым галсом, всегда уступает дорогу судну, идущему правым галсом

Судно, находящящееся на ветре, уступает дорогу судну, находящемуся под ветром

Обгонять судно можно лишь тогда, когда вы находитесь на виду друг у друга
Судно, которое находится с курсового угла более 112.5\degree считается обгоняющим
У обгоняемого судна ночью виден только кормовой огонь

Считается завершенным обгон, когда вы окончательно обогнали судно, и оставили его позади.

В дневное время огни судна заменяются шарами, конусами и корзинами, ромбами

Если нужно пересечь курс судна с носа, то пересекаем его под углом в 90\degree

Федеральное Государственное управление волго-балтийское бассейного 
управления Водных Путей и судоходства

1. Минестерство транспорта
2. Федеральное агентство морского и речного судоходства
3. ФГУ

Волгобалт охватывает:
	Северо-западный регион
	Ленинградская область
	Новгородская область
	Вологодская область
	Псковская область
	Калининградская область
	Санкт-Петербург
	Карелия (частично)
	Беломоро-балтийский канал

4900км водных путей эксплуатации
11 высоконапорных шлюзов
3 ГЭС
4 водосброса
25 земляных плотин и дамб
9 мостовых переходов
12 паромных переправ
8 маяков, к которым относятся Осиновецкий (70м), Сторожецкий, Свирский (по указу Петра I)


Река Нева --- длина 74км от истока -- Ладожское озеро
Площадь бассейна -- 5км^2
Ширина - 1000м
Самое узкое 200м в районе Ивановских порогов
Глубина - минимально 4м, максимально 24м

Впадает 26 рек, Нева имеет 40 островов.
Скорость Невы 0.8-1.01 м/с
Расход воды 116m^3/c	
Средний расход 2500 м/с

Имеет пороги
	Константиновские - в районе поселка Торопово
	Ивановские
	Перекат Кривое Колено
	Кошкинский фарватер - выход из невы в Ладогу. Имеет 6 колен. Начало от Бугровского приёмного буя, конец -- напротив Преображенской горы.

Пятое колено имеет свальное течение на Шереметьевскую отмель
Кривое колено -- находится у Невского лесопарка и Усть-Славянки в 30 километрах от устья. Имеет 3 крутых поворота реки.

У Новосаратовки --- к правому берегу, у причала Невский лесопарк -- к левому.

В среднем ширина 500-700 метров

Мосты Санкт-Петербурга
	Благовещенский
	Дворцовый
	Троицкий
	Литейный
	Большеохтинский
	Александра Невского
	Финлядский
	Володарский
	Вантовый(Большой Обуховский)

Разделение река-море происходит по Благовещенскому мосту

Нева разделяется на
	Малая Невка --- от Большой Невы у стрелки Каменного острова. Длина 4.9км, ширина 120-300 
		Глубина до 6.8м
	
	Малая Нева --- от стрелки Васильевского Острова, имеет два моста. Биржевой и Тучков мост
		длина 4250м, 
		ширина 200-400м
		глубина 3.7м
Сайт ФГПУ Волгобалта
	Расписание радиосвязи.

\section{Мариинская система}
Волго-балтийский канал начинается от Онеги, и заканчивается на Рыбинском водохранилище. Протяженность Волго-Балтийского канала --- 368км.

Волго-балтийский канал имеет крутой северный (балтийский) и пологий южный (каспийский) склон.
Северный склон имеет 6 шлюзов, и осуществляется подъем до 80м.

Шлюзы
	Шлюз 1. Ватигорский гидроузел
	Шлюз 2. Белоусовский
	Шлюз 3,4,5. Новинский
	Шлюз 6. Пахимовский

Размеры 270x18м.
От шлюза 6 до Череповецкого узла идёт водоразделительный блеф.

Каналы связи на реке
Осуществляются на УКВ.
	5 канал -- вызов судов, согласование маневров и 
		сигналов бедствия --- всегда должен быть включен
	2 канал -- связь между судами
	3 канал -- связь с диспетчером шлюзов
	4 канал -- связь с другими службами речфлота
	25, 43 каналы -- связь между яхтами

За 1.5 часа до подхода к шлюзам --- выход на связь с диспетчером, и получаем данные о судопропуске.
За 1 час получаем подтверждение о судопропуске. Очередь прохождение за таким-то судном
Зашли в шлюз. Накидываем швартовы на гак. 

Беломоро-балтийский канал начинается от п. Павенец до г. Беломорск.
Имеет 19 гидроузлов
	7 гидроузлов южного склона, напор -- 69 метров
	12 гидроузлов северного склона, напор -- 103 метра

Всего имеет 32 камеры шлюзов

Имеет 15 плотин, 5 ГЭС, 12 водосбросов, длительность навигации --- 165 суток. Габариты шлюзов 135x14.3м. Габариты судового хода
	глубина 4м
	ширина 36м
	радиус закругления на судовом ходе 500м 
	скорость при прохождении не более 8 км/ч

\end{document}               

