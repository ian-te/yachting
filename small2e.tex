% Everything to the right of a  %  is a remark to you and is ignored by LaTeX.

% The Local Guide tells how to run LaTeX.

% WARNING!  Do not type any of the following 10 characters except as directed:
%                &   $   #   %   _   {   }   ^   ~   \   

\documentclass{article}        % Your input file must contain these two lines 
\usepackage[utf8]{inputenc}
\usepackage[russian]{babel}
\begin{document}               % plus the \end{document} command at the end.




\section{Устройство паруса}          
Парус --- движитель. Парус делается из парусины, из любого материала (японцы ставили паруса из плексигласса)

Устройство паруса
	Фаловый угол \\
	Фаловая дощечка для жёсткости \\
	Передняя шкаторина
	Задняя шкаторина
	Нижняя шкаторина
	Швы
	Латкарманы -- туда вставляются доски, либо длинные куски пластмассы, чтобы парус держал форму
	Углы (боуты)
	Шкотовый узел
	Галзсовый узел
	Рифы
	Ползуны -- продеваются в ликтроз

\subsection{Виды парусов}
Прямые паруса
Косые паруса
	латинские
	гафельные
	клевера
	стаксели

Прямой парус имеет прямоугольную форму или трапецевидную форму, угол между ветром и парусом может составлять 67\deg

Косые паруса нужны для того, чтоб ходить против ветра

Латинский парус
Парус треугольной формы, привязывают к мачте или рее длинной стороной вдоль диаметральной плоскости судна по направлению к корме и растягивают его при помощи шкота

Гафельный парус
Парус трапецевидной формы, разделяется на трисселя, люгерные, рейковые

Форма
Неправильная трапеция

Крепление
Верхняя часть крепится к гафелю бизани, 
Нижняя часть к бизань гику
Вертикальная сторона крепится к мачте

Клевера
Парус треугольной формы ставят между фок-мачтой и бушпритом

\subsection{Курсы парусов}

\subsection{Узлы}

\end{document}                 % The input file ends with this command.

