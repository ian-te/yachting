% Everything to the right of a  %  is a remark to you and is ignored by LaTeX.

% The Local Guide tells how to run LaTeX.

% WARNING!  Do not type any of the following 10 characters except as directed:
%                &   $   #   %   _   {   }   ^   ~   \   

\documentclass{article}        % Your input file must contain these two lines 
\usepackage[utf8]{inputenc}
\usepackage[russian]{babel}
\begin{document}               % plus the \end{document} command at the end.




\section{Устройство паруса}          
Парус --- движитель. Парус делается из парусины, из любого материала (японцы ставили паруса из плексигласса)

Устройство паруса
	Фаловый угол \\
	Фаловая дощечка для жёсткости \\
	Передняя шкаторина
	Задняя шкаторина
	Нижняя шкаторина
	Швы
	Латкарманы -- туда вставляются доски, либо длинные куски пластмассы, чтобы парус держал форму
	Углы (боуты)
	Шкотовый узел
	Галзсовый узел
	Рифы
	Ползуны -- продеваются в ликтроз

\subsection{Виды парусов}
Прямые паруса
Косые паруса
	латинские
	гафельные
	клевера
	стаксели

Прямой парус имеет прямоугольную форму или трапецевидную форму, угол между ветром и парусом может составлять $67\deg$

Косые паруса нужны для того, чтоб ходить против ветра

Латинский парус
Парус треугольной формы, привязывают к мачте или рее длинной стороной вдоль диаметральной плоскости судна по направлению к корме и растягивают его при помощи шкота

Гафельный парус
Парус трапецевидной формы, разделяется на трисселя, люгерные, рейковые

Форма
Неправильная трапеция

Крепление
Верхняя часть крепится к гафелю бизани, 
Нижняя часть к бизань гику
Вертикальная сторона крепится к мачте

Клевера
Парус треугольной формы ставят между фок-мачтой и бушпритом


Стаксель
Парус треугольной формы (на яхте он передний).

Генакер
Парус с высоким шкотовым углом, его галсовый угол крепится к палубе на носу яхты или бушприту.

Спинакер
Крепится с помощью спинакер-гика

Мачта состоит из 
Шпора
Верхняя часть мачты называется топ

Бушприт состоит из нока(верхняя часть) и шпор (нижняя часть)

Гик - состоит из пятки и нока


Рея -- горизонтальное рангоутное дерево, подвешенное за середину при помощи бортов и бейфута к мачте или стенге.

Стеньга -- рангоутное дерево, первое удлиннение нижних мачт

Краспица
Одно из основных средств распорка между мачтой стоячего такелажа



Такелаж
Бегучий

Стоячий
Поддерживать и укреплять рангоут судна
К нему относятся тросы трёх типов:
ванты
Штаги
Ватербак штаги
Стеньштаг
Топванты
Бакштаг
Ватерштаг

Бегучий такелаж
Снасти служат для уборки и постановки парусов

К ним относятся
Фалы
Брассы
Шкоты
Галсы
Булини
Гордени
Гитовы
Тапенанты


Рангоутное дерево
Гафель
Реек
Выстрел
Утлегарь
Бушприт
Рей
Бранстеньга


\subsection{Снасти}
Штаг
Стоячий такелаж, ужераживает мачту с носа

Ахтерштаг
Стоячий такелаж, удерживает мачту с корммы

Бакштаг
Стоячий такелаж, удерживает мачту с борта и с кормы

Топенант
Снасть такелажа, поддерживает рангоутное дерево

ФАл
Бегучий такелаж, служит для подъема паруса

Нирал
Снасть бегучего такелажа, служит для спуска паруса

Шкот
Снасть бегучего такелажа, служит для управления парусом

Гитов
Бегучий такелаж, по направлению действия противоположный шкоту

Ванта
Стоячий такелаж, удерживает мачту в плоскости шпангоута

Галс
Бегучий такелаж, крепит нижний угол косого паруса

Нирал
Бегучий такелаж, направление которого противоположно фалу (для спуска паруса)


\subsection{Курсы парусов}

\subsection{Узлы}

\end{document}                 % The input file ends with this command.

